%:
%\documentclass[12pt,mythesisstyle,twoside]{report}%% for two-sided printing
\documentclass[12pt,mythesisstyle]{report}
%\usepackage{files/mythesisstyle}
%\usepackage{bbm}

\usepackage{amssymb,amsmath}
\usepackage{calc}
\usepackage{verbatim}
%\usepackage[mathscr]{eucal}
%\usepackage{pstricks}

% \textbf{\textit{A}}

\title{Data And Uncertainity\\Assignment 8 November 2017}
\author{Leonardo Ripoli}
\date{}

\begin{document}

\begin{titlepage}
\maketitle
\end{titlepage}
\textbf{Ex1.1)}\\
Let \(\Omega=\mathbb{N}\), show that \textbf{\textit{A}}, the set of all cofinite subsets in \(\Omega\), forms an algebra.
\\
\\The proof is done by checking that the 3 properties of an algebra are satisfied: empty set, closeness to complement and closeness to finite unions.
\\
\begin{enumerate}
\item \(\varnothing\in\textbf{\textit{A}}\) trivially because it is finite.\\
\item By definition if \(A\in\textbf{\textit{A}}\) it is cofinite, therefore either it is finite and its complement is infinite, or the other way around: in either case \(A\in\textbf{\textit{A}}\implies A^\complement\in\textbf{\textit{A}}\).\\
\item Let \(A_1, A_2, A_3...\in\textbf{\textit{A}}\), then if all of the \(A_i\) are finite, clearly \(\cup_{k=1}^nA_i\) is finite and therefore \(\cup_{k=1}^nA_i\in\textbf{\textit{A}}\). If at least one of the \(A_i\) is infinite, then \(\cup_{k=1}^nA_i\) is infinite, we have to show that \((\cup_{k=1}^nA_i)^\complement\) is finite. Let's say at least \(A_k\) is infinite (and therefore \(A_k^\complement\) is finite because \(A_k\) is cofinite by hypothesis), we have that \((\cup_{k=1}^nA_i)^\complement=\cap_{k=1}^nA_i^\complement\), and since \(\cap_{k=1}^nA_i^\complement\subseteq A_k^\complement\) and \(A_k^\complement\) is finite, it must be that \(\cap_{k=1}^nA_i^\complement\) is finite and therefore we have shown that \(\cup_{k=1}^nA_i\in\textbf{\textit{A}}\).
\end{enumerate}

\textbf{Ex1.2)}\\
Let \textbf{\textit{A}} be the algebra of cofinite sets and define the set function \(\mu(A)=1\) if A is finite, and 0 otherwise: show that \(\mu\) is normalised and additive.
\begin{enumerate}
\item \(\mu(\mathbb{N})=1\) since \(\mathbb{N}\) is infinite, therefore \(\mu\) is normalised.\\
\end{enumerate}


\end{document}
