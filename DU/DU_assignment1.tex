%:
%\documentclass[12pt,mythesisstyle,twoside]{report}%% for two-sided printing
\documentclass[12pt,mythesisstyle]{report}
%\usepackage{files/mythesisstyle}
\usepackage{bbm}

\usepackage{amssymb,amsmath}
\usepackage{calc}
\usepackage{verbatim}
%\usepackage[mathscr]{eucal}
%\usepackage{pstricks}

% \textbf{\textit{A}}

\title{Data And Uncertainity\\Assignment 8 November 2017}
\author{Leonardo Ripoli}
\date{}

\begin{document}

\begin{titlepage}
\maketitle
\end{titlepage}
\textbf{Ex1.1)}\\
Let \(\Omega=\mathbb{N}\), show that \textbf{\textit{A}}, the set of all cofinite subsets in \(\Omega\), forms an algebra.
\\
\\The proof is done by checking that the 3 properties of an algebra are satisfied: empty set, closeness to complement and closeness to finite unions.
\\
\begin{enumerate}
\item \(\varnothing\in\textbf{\textit{A}}\) trivially because it is finite.\\
\item By definition if \(A\in\textbf{\textit{A}}\) it is cofinite, therefore either it is finite and its complement is infinite, or the other way around: in either case \(A\in\textbf{\textit{A}}\implies A^\complement\in\textbf{\textit{A}}\).\\
\item Let \(A_1, A_2, A_3...\in\textbf{\textit{A}}\), then if all of the \(A_i\) are finite, clearly \(\cup_{k=1}^nA_i\) is finite and therefore \(\cup_{k=1}^nA_i\in\textbf{\textit{A}}\). If at least one of the \(A_i\) is infinite, then \(\cup_{k=1}^nA_i\) is infinite, we have to show that \((\cup_{k=1}^nA_i)^\complement\) is finite. Let's say at least \(A_k\) is infinite (and therefore \(A_k^\complement\) is finite because \(A_k\) is cofinite by hypothesis), we have that \((\cup_{k=1}^nA_i)^\complement=\cap_{k=1}^nA_i^\complement\), and since \(\cap_{k=1}^nA_i^\complement\subseteq A_k^\complement\) and \(A_k^\complement\) is finite, it must be that \(\cap_{k=1}^nA_i^\complement\) is finite and therefore we have shown that \(\cup_{k=1}^nA_i\in\textbf{\textit{A}}\).
\end{enumerate}
\textbf{Ex1.2)}\\
Let \textbf{\textit{A}} be the algebra of cofinite sets and define the set function \(\mu(A)=1\) if A is finite, and 0 otherwise: show that \(\mu\) is normalised and additive.
\begin{enumerate}
\item \(\mu(\mathbb{N})=1\) since \(\mathbb{N}\) is infinite, therefore \(\mu\) is normalised.
\item For additivity we have to show that \(\mu(\cup_{k=1}^nA_i)=\sum_{k=1}^n\mu(A_i), \forall A_k\in\textbf{\textit{A}}\) s.t. \(A_i\cap A_j=\varnothing\) \(i\neq j\). If all \(A_k\) are finite, then \(\cup_{k=1}^nA_i\) is finite and therefore \(\mu(\cup_{k=1}^nA_i)=0=\sum_{k=1}^n\mu(A_i)\), since \(\mu(A_k)=0\) \(\forall k\). To address the case where at least one of the \(A_k\) is infinite, firstly we show that the condition \(A_1, A_2\in\textbf{\textit{A}}, A_1\cap A_2=\varnothing, A_1\cup A_2\) is infinite means that only one between \(A_1\) and \(A_2\) can be infinite, and not both. In fact, if we assume that \(A_1\) and \(A_2\) are both infinite and that \(A_1\cap A_2=\varnothing\), we have that \(A_1^\complement\) and \(A_2^\complement\) are both finite (\(A_1\) and \(A_2\) being cofinite sets) and therefore we would have: \(A_1\cap A_2=\varnothing\implies A_1^\complement\cup A_2^\complement=\mathbb{N}\) and this is an absurd as the union of two finite sets would give the whole set of natural numbers which is infinite. By applying the previous result to each pair \(A_i\cap A_j=\varnothing,  i\neq j\), we know that the condition \(A_k\in\textbf{\textit{A}}\) \(\forall k=1,..n,A_i\cap A_j=\varnothing\) \(i\neq j,\cup_{k=1}^nA_i\) is infinite, implies that only one of the \(A_k\) can be infinite and therefore we have that \(\mu(\cup_{k=1}^nA_i)=1\) since \(\cup_{k=1}^nA_i\) is infinite, and \(\sum_{k=1}^n\mu(A_i)=1\) because only one of the k terms of the sum will be 1 (the contribution from the only infinite set) and all other contributions will be zero. Therefore we have shown that \(\mu(\cup_{k=1}^nA_i)=\sum_{k=1}^n\mu(A_i), \forall A_k\in\textbf{\textit{A}}\).
\end{enumerate}
\textbf{Ex1.3)}\\
Show that \(\mu\) is not continuous at the empty set.\\\\
We take $A_k=\mathbb{N}\setminus\{0,1,...k\}$, all $A_k$ are cofinite, in fact $A_k^\complement=\{0,1,...k\}$, a finite set, and in addition, by construction, $\cap_{k\in\mathbb{N}}=\varnothing$ and $A_1\supset A_2...$. We have $\mu(A_k)=\mu(\mathbb{N})-\mu(\{0,1,...k\})=1$, and therefore $\lim_{k\rightarrow\infty}\mu(A_k)=\lim_{k\rightarrow\infty}1=1$.
\\\\
\textbf{Ex2.1)}\\
Let $(\Omega, \textbf{\textit{A}}, P)$ be a probability space. Let f be a non negative random variable, and suppose that $\int fdP=1$, on \textbf{\textit{A}} define the set function F by $F(A)=\int1_A\cdot fdP$.
\\
Show that F is a probability on $(\Omega, \textbf{\textit{A}})$.
\\
\\The proof is done by checking the properties of a probability on a sigma algebra: normalisation and sigma additivity. Sigma (infinite) additivity proves both simple (non infinite) additivity and continuity at $\varnothing$ (result used without proof from the lecture notes: exercise 2.2, point 2).

\begin{enumerate}
  \item $F(\Omega)=\int 1_\Omega\cdot fdP$, remembering the definition of the indicator function, we have that $F(\Omega)=\int 1_\Omega\cdot fdP=\int_\Omega fdP=1$ by hypothesis.
  \item Let $A_1, A_2,...\in\textbf{\textit{A}}$, $A_i\cap A_j=\varnothing$ $i\neq j$. We define $C=\cup_{k=1}^{\infty}A_k$, $B_n=\cup_{k=1}^{n}A_k$ we know both $C,B\subseteq\textbf{\textit{A}}$ since \textbf{\textit{A}} is a sigma algebra and therefore closed under infinite unions. Define $g_n=f\cdot 1_{B_n}$, $g_n$ is measurable since f is measurable by hypothesis and the indicator function is measurable too and the product of measurable functions is measurable. Define $g=f\cdot 1_{C}$. Since $f\geq 0$, we have that $0 \leq g_1\leq g_2...\leq g$, in fact $f\cdot 1_{B_n}\leq f\cdot 1_{B_{n+1}}\leq f\cdot 1_C$ as $B_n\subseteq B_{n+1}...\subseteq C$. In addition, by construction, $g_n\uparrow g$ as $n\rightarrow \infty$. We are therefore in the monotone convergence theorem hypotheses, and we can say that: $F(C)=\int_{\Omega} f\cdot 1_{C}dP=\int_{\Omega} gdP=\lim{n \rightarrow \infty}\int_{\Omega} g_n dP$ (the last equality comes from the Monotone Convergence Theorem). Remembering that $g_n=f\cdot 1_{B_n}$ and that $B_n=\cup_{k=1}^{n}A_k$ where all of the $A_k$ are mutually disjoint, we have that $g_n=f\cdot 1_{B_n}=\sum_{k=1}^n f\cdot 1_{A_k}$, and therefore $F(\cup_{k=1}^{n}A_k)=\int_{\Omega} g_n dP=\int\sum_{k=1}^n f\cdot 1_{A_k}dP=\sum_{k=1}^n\int f\cdot 1_{A_k}dP=\sum_{k=1}^n F(A_k)$. And it follows from previous results that $F(C)=F(\cup_{k=1}^{\infty}A_k)=\int_{\Omega} f\cdot 1_{C}dP=\int_{\Omega} gdP=\lim{n \rightarrow \infty}\int_{\Omega} g_n dP=\lim{n \rightarrow \infty} \sum_{k=1}^n F(A_k)=\sum_{k=1}^{\infty} F(A_k)$. And this proves the sigma additivity.
\end{enumerate}

\textbf{Ex2.2)}\\
Show that $P(A)=0\implies F(A)=0$.
\begin{itemize}
\item We start by proving for simple f. Let $f=\sum_{j=1}^p f_j\cdot 1_{B_j}$, then we have $F(A)=\int f\cdot 1_{A}dP=\int \sum_{j=1}^p f_j\cdot 1_A\cdot 1_{B_j}dP=\sum_{j=1}^p \int f_j\cdot 1_{A\cap B_j}dP=\sum_{j=1}^p f_jP(A\cap B_j)=0$, the zero equality coming from the fact that we have $P(A)=0$ by hypothesis and $A\cap B_j\subseteq A$ therefore it must be $0<=P(A\cap B_j)<=P(A)=0$. We have therefore proved that that for f simple $P(A)=0\implies F=0$.
\item We now consider the case of a positive function that is not simple. We consider $f_n=\sum_{j=1}^{p_n} f_{jn}\cdot 1_{B_jn}$, with $\{f_n\}$ an increasing sequence of simple functions s.t. $f_n\uparrow f$, and therefore (thanks to the Monotone Convergence Theorem) $\int f_ndP\uparrow\int fdP$ (an example of the construction of an increasing sequence of simple functions whose limit is f is in the lecture notes, in the paragraph "The integral"). Note that the results $f_n\uparrow f$ and $\int f_ndP\uparrow\int fdP$ hold also for $f_n\cdot 1_A\uparrow f\cdot 1_A$ and $\int f_n\cdot 1_A dP\uparrow\int f\cdot 1_A dP$, in fact we could define $g_n=f_n\cdot 1_A$ and $g=f\cdot 1_A$ and we would have $g_n\uparrow g$ and $\int_{\Omega} g_ndP\uparrow\int_{\Omega} gdP$ since we are in the Monotone Convergence Theorem hypotheses. We have demonstrated in the first part of this exercise 2.2 that for a simple function, if $P(A)=0$ the integral will always be zero, therefore it will be $\int f_n\cdot 1_{A}dP=\int \sum_{j=1}^{p_n} f_{jn}\cdot 1_A\cdot 1_{B_{jn}}dP=\sum_{j=1}^{p_n} f_jP(A\cap B_{jn})=0$, therefore $\int f_n\cdot 1_{A}dP=0$ $\forall n$, and $0=\lim_{n \to \infty}\int f_n\cdot 1_{A}dP=\int f\cdot 1_{A}dP$, the last equality coming from the Monotone Convergence Theorem, and being $\int f\cdot 1_{A}dP=F(A)$, we have demonstrated that $P(A)=0\implies F(A)=0$.
\end{itemize}

\textbf{Ex2.3)}\\
Show that $\int gdP=F=\int fdP\implies f=g$ a.e. ($\Longleftarrow$ is trivial).
\\\\
We can use the result that for $z\geq 0$, $\int zdP=0 \implies z=0$ a.e. We have by hypothesis that $\int (f-g)dP=0$, let $h=f-g$, using $h_+=$max$\{f,0\}$ and $h_-=$min$\{f,0\}$, we have $h=h_+-h_-$. We consider $\Omega_1=\{x\in\Omega \mid f\geq g\}$ and $\Omega_2=\{x\in\Omega \mid f<g\}$, it's $\Omega_1\cap\Omega_2=\varnothing$ and $\Omega_1\cup\Omega_2=\Omega$. Since the property $\int_A hdP=0$ holds on any $A\subseteq\Omega$, we have $\int_{\Omega_1} hdP=0$ and $\int_{\Omega_2} hdP=0$. So, using definitions of $h_+$ and $h_-$, we have $\int_{\Omega_1} hdP=\int_{\Omega_1} h_+dP=0$ and $\int_{\Omega_2} hdP=-\int_{\Omega_2} h_-dP=0\implies\int_{\Omega_2} h_-dP=0$: in both cases we can use the result $z\geq 0$, $\int zdP=0 \implies z=0$ a.e. and therefore we have $f=g$ a.e. on $\Omega_1\cup\Omega_2=\Omega$.
\\\\
\textbf{Ex3.1)}\\
Show that $A_1, A_2,...\in\textbf{\textit{A}}$, $A_1\supset A_2...$ and $P(A_k)\rightarrow 0$ as $k\rightarrow\infty$, $f\geq 0$, then $\int_{A_k}fdp\rightarrow0$ for $k\rightarrow\infty$.
\\\\
\\Let $f_n(x)=\bigg\{
\begin{array}{ll}
n \ \ \ for \ x \ \mid f(x)\geq n\\
f(x) \ \ \ for \  x \ \mid 0\leq f(x)<n\\
\end{array}$
\\
Proof\\
We have that $\{f_n\}$ is an increasing sequence of positive measurable functions (since f is measurable), and, by construction, $f_n \uparrow f$ pointwise. For any $A_k\in\textbf{\textit{A}}$, we have that
\begin{equation} \label{ex3_1}
\int_{A_k}f dP=\int_{A_k}(f-f_n) dP+\int_{A_k}f_n dP
\end{equation}
$\int_{A_k}f dP=\int_{A_k}(f-f_n) dP+\int_{A_k}f_n dP$. Since $f_n \uparrow f$ and $0\leq f_n\leq f$, $(f-f_n)$ is a positive quantity, and we have that $\int_{A_k}(f-f_n) dP\leq\int_{\Omega}(f-f_n)dP$ because we consider a bigger domain of integration for a positive quantity. In addition, since by construction $f_n\leq n$, it's $\int_{A_k}f_n dP\leq nP(A_k)$. By monotone convergence theorem, for n sufficiently high, $\forall \epsilon$ we can have $\int_{\Omega}(f-f_n)dP\leq \frac{\epsilon}{2}$, and also, being by hypothesis $P(A_k)\rightarrow 0$, for k sufficiently high we can have $P(A_k)\leq\frac{\epsilon}{2n}$, therefore in (\ref{ex3_1}) we have that $\int_{A_k}f dP\leq \frac{\epsilon}{2}+\frac{\epsilon}{2n}$, therefore, ultimately, $P(A_k)\rightarrow 0\implies\int_{A_k}f dP\rightarrow 0$.
\\
\\
\textbf{Ex3.2)}\\
Show that $A_1, A_2,...\in\textbf{\textit{A}}$, $\sum_{k=1}^{\infty} P(A_k)<\infty \implies P(\cup_{l=k}^{\infty}A_l)\rightarrow 0$ if $k\rightarrow\infty$. Let $\sum_{k=1}^{\infty} P(A_k)=z\geq0$, from sigma additivity, $P(\cup_{l=k}^{\infty}A_l)=\sum_{l=k}^{\infty}P(A_k)$. From the convergence of the series, we have that $\sum_{l=k}^{\infty}P(A_k)=(z-\sum_{l=k}^{\infty}P(A_k))\rightarrow 0$ as $k\rightarrow\infty$, and this proves that $P(\cup_{l=k}^{\infty}A_l)\rightarrow 0$ if $k\rightarrow\infty$.

\end{document}

