%:
%\documentclass[12pt,mythesisstyle,twoside]{report}%% for two-sided printing
\documentclass[12pt,mythesisstyle]{report}
%\usepackage{files/mythesisstyle}
\usepackage{bbm}

\usepackage{amssymb,amsmath}
\usepackage{calc}
\usepackage{verbatim}
%\usepackage[mathscr]{eucal}
%\usepackage{pstricks}

% \textbf{\textit{A}}

\title{Data And Uncertainity\\Assignment 8 November 2017}
\author{Leonardo Ripoli}
\date{}

\begin{document}

\begin{titlepage}
\maketitle
\end{titlepage}
\textbf{Ex1.1)}\\
Let \(\Omega=\mathbb{N}\), show that \textbf{\textit{A}}, the set of all cofinite subsets in \(\Omega\), forms an algebra.
\\
\\The proof is done by checking that the 3 properties of an algebra are satisfied: empty set, closeness to complement and closeness to finite unions.
\\
\begin{enumerate}
\item \(\varnothing\in\textbf{\textit{A}}\) trivially because it is finite.\\
\item By definition if \(A\in\textbf{\textit{A}}\) it is cofinite, therefore either it is finite and its complement is infinite, or the other way around: in either case \(A\in\textbf{\textit{A}}\implies A^\complement\in\textbf{\textit{A}}\).\\
\item Let \(A_1, A_2, A_3...\in\textbf{\textit{A}}\), then if all of the \(A_i\) are finite, clearly \(\cup_{k=1}^nA_i\) is finite and therefore \(\cup_{k=1}^nA_i\in\textbf{\textit{A}}\). If at least one of the \(A_i\) is infinite, then \(\cup_{k=1}^nA_i\) is infinite, we have to show that \((\cup_{k=1}^nA_i)^\complement\) is finite. Let's say at least \(A_k\) is infinite (and therefore \(A_k^\complement\) is finite because \(A_k\) is cofinite by hypothesis), we have that \((\cup_{k=1}^nA_i)^\complement=\cap_{k=1}^nA_i^\complement\), and since \(\cap_{k=1}^nA_i^\complement\subseteq A_k^\complement\) and \(A_k^\complement\) is finite, it must be that \(\cap_{k=1}^nA_i^\complement\) is finite and therefore we have shown that \(\cup_{k=1}^nA_i\in\textbf{\textit{A}}\).
\end{enumerate}
\textbf{Ex1.2)}\\
Let \textbf{\textit{A}} be the algebra of cofinite sets and define the set function \(\mu(A)=1\) if A is finite, and 0 otherwise: show that \(\mu\) is normalised and additive.
\begin{enumerate}
\item \(\mu(\mathbb{N})=1\) since \(\mathbb{N}\) is infinite, therefore \(\mu\) is normalised.\\
\item For additivity we have to show that \(\mu(\cup_{k=1}^nA_i)=\sum_{k=1}^n\mu(A_i), \forall A_k\in\textbf{\textit{A}}\) s.t. \(A_i\cap A_j=\varnothing\) \(i\neq j\). If all \(A_k\) are finite, then \(\cup_{k=1}^nA_i\) is finite and therefore \(\mu(\cup_{k=1}^nA_i)=0=\sum_{k=1}^n\mu(A_i)\), since \(\mu(A_k)=0\) \(\forall k\). To address the case where at least one of the \(A_k\) is infinite, firstly we show that the condition \(A_1, A_2\in\textbf{\textit{A}}, A_1\cap A_2=\varnothing, A_1\cup A_2\) is infinite means that only one between \(A_1\) and \(A_2\) can be infinite, and not both. In fact, if we assume that \(A_1\) and \(A_2\) are both infinite and that \(A_1\cap A_2=\varnothing\), we have that \(A_1^\complement\) and \(A_2^\complement\) are both finite (\(A_1\) and \(A_2\) being cofinite sets) and therefore we would have: \(A_1\cap A_2=\varnothing\implies A_1^\complement\cup A_2^\complement=\mathbb{N}\) and this is an absurd as the union of two finite sets would give the whole set of natural numbers which is infinite. By applying the previous result to each pair \(A_i\cap A_j=\varnothing,  i\neq j\), we know that the condition \(A_k\in\textbf{\textit{A}}\) \(\forall k=1,..n,A_i\cap A_j=\varnothing\) \(i\neq j,\cup_{k=1}^nA_i\) is infinite, implies that only one of the \(A_k\) can be infinite and therefore we have that \(\mu(\cup_{k=1}^nA_i)=1\) since \(\cup_{k=1}^nA_i\) is infinite, and \(\sum_{k=1}^n\mu(A_i)=1\) because only one of the k terms of the sum will be 1 (the contribution from the only infinite set) and all other contributions will be zero. Therefore we have shown that \(\mu(\cup_{k=1}^nA_i)=\sum_{k=1}^n\mu(A_i), \forall A_k\in\textbf{\textit{A}}\).\\
\end{enumerate}
\textbf{Ex1.3)}\\
Show that \(\mu\) is not continuous at the empty set.\\\\
We take $A_k=\mathbb{N}\setminus\{0,1,...k\}$, all $A_k$ are cofinite, in fact $A_k^\complement=\{0,1,...k\}$, a finite set, and in addition, by construction, $\cap_{k\in\mathbb{N}}=\varnothing$ and $A_1\supset A_2...$. We have $\mu(A_k)=1$ as $A_k$ is infinite $\forall$ k.

We Let's fix n and consider the set \(B_n=\{0,2,...2^{n-1}\}\), and put \(A_k=B_n\setminus\{0,2,...,2^{k-1}\}\), \(k=0,...n\): clearly \(A_k\in\textbf{\textit{A}}\) in fact \(A_k\subset\mathbb{N}\), \(k=0,...n\) and is finite. In addition, by construction we have that \(A_1\supset A_2...\supset A_n\) and $\cap_{j}^n A_j=\varnothing$. As \(n\rightarrow\infty\) $A_n\rightarrow\mathbb{N}\setminus\{0,2,4,...\}$, i.e. to the set of odd numbers which is infinite and would therefore have measure 1. The set of cofinite sets is not a sigma algebra, in fact if we consider the \(\cup_{n=1}^\infty B_n=C\) where $B_n=\{0,2,...2^{n-1}\}$, we have that C is the set of odd numbers, which is not cofinite as it is infinite and its complement (i.e. the set of even numbers) is infinite as well.
\\\\
\textbf{Ex2.1)}\\
Let $(\Omega, \textbf{\textit{A}}, P)$ be a probability space. Let f be a non negative random variable, and suppose that $\int fdP=1$, on \textbf{\textit{A}} define the set function F by $F(A)=\int1_A\cdot fdP$.
\\\\
Show that F is a probability on $(\Omega, \textbf{\textit{A}})$.
\\
\\The proof is done by checking that the 3 properties of a probability: normalisation, additivity and continuity at $\varnothing$ (this latter is proved using the equivalent sigma additivity).
\\
\begin{enumerate}
\item $F(\Omega)=\int 1_\Omega\cdot fdP$, remembering the definition of the indicator function, we have that $F(\Omega)=\int 1_\Omega\cdot fdP=\int_\Omega fdP=1$.\\
\item Let $A_1, A_2,...A_n\in\textbf{\textit{A}}$, with $A_i\cap A_j=\varnothing$ for $i\neq j$. $\sum_{k=1}^n F(A_k)=\sum_{k=1}^n\int 1_{A_k}\cdot fdP=\sum_{k=1}^n\int_{A_k}\cdot fdP$. On the other hand, $F(\cup_{k=1}^n A_k)=\int 1_{\cup_{k=1}^n A_k}\cdot fdP=\int_{\cup_{k=1}^n A_k}fdP$. Since all the domains $A_k$ are disjoint we have that $\int_{\cup_{k=1}^n A_k}fdP=\sum_{k=1}^n \int_{A_k}fdP$, and therefore $\sum_{k=1}^n F(A_k)=\sum_{k=1}^n\int 1_{A_k}\cdot fdP=F(\cup_{k=1}^n A_k)$, which proves the additivity.\\
\item Let $A_1, A_2,...A_n\in\textbf{\textit{A}}$, with $A_i\cap A_j=\varnothing$ for $i\neq j$. $\sum_{k}P(A_k)=\sum_{k}F(A_k)=\sum_{k}\int 1_{A_k}\cdot fdP$. Using monotone convergence theorem, we know we can build $f_n\uparrow f$ with $f_n$ simple functions and so that $\int f_n dP\uparrow \int fdP$. It's $\displaystyle{\sum_{k}\int 1_{A_k}\cdot fdP=\lim_{n \to \infty}\sum_{k}\int 1_{A_k}\cdot f_ndP}$ (in fact $\int_{A_k} f_ndP\uparrow\int_{A_k} fdP \forall A_k$). Expressing explicitly $f_n$ as simple function we have: $f_n=\sum_{j=1}f_{n_j}\cdot 1_{B_{nj}}$. \underline{In the following expressions, for simplicity of notation, we will omit the}\\\underline{$\lim_{n \to \infty}$}\underline{or $\uparrow$ whenever there is $f_n$, but it is intended each time there is $f_n$}. So expressing in terms of $f_n$, $F(A_k)=\int\sum_j f_{n_j}\cdot 1_{B_{nj}}\cdot 1_{A_k}dP=\int\sum_j f_{n_j}\cdot 1_{B_{nj} \cap {A_k}}dP$, and therefore $\sum_k F(A_k)=\sum_k \int\sum_j f_{n_j}\cdot 1_{B_{nj} \cap {A_k}}dP=\\=\sum_k\sum_j f_{n_j}\mu(B_{nj} \cap {A_k})$. On the other hand,\\$F(\cup_k A_k)=\int \sum_j f_{n_j}1_{B_{nj}} \cdot 1_{\cup _k {A_k}}dP=\int \sum_j f_{n_j}1_{B_{nj}\cap (\cup _k {A_k})}dP=\\=\sum_j f_{n_j}\mu({B_{nj}\cap (\cup _k {A_k})})=\sum_j f_{n_j}\mu(\cup_k({B_{nj}\cap {A_k}}))$. From the sigma additivity property of $\mu$, and since $A_i\cap A_j=\varnothing$ for $i\neq j$ $\implies ({B_{nj}\cap {A_k}})\cap({B_{nj}\cap {A_z}})=\varnothing$ for $k\neq z$, we have that $\mu(\cup_k({B_{nj}\cap {A_k}}))=\sum_k\mu({B_{nj}\cap {A_k}})$.\\
Putting all together we have that $\sum_k F(A_k)=\sum_k\sum_j f_{n_j}\mu(B_{nj} \cap {A_k})=F(\cup_k A_k).$
\\\\
\textbf{Ex2.2)}\\
Show that $P(A)=0\implies F(A)=0$.
\\
\\We start by proving for simple f. Let $f=\sum_{j=1}^p f_j\cdot 1_{B_j}$, then we have $F(A)=\int f\cdot 1_{A}dP=\int \sum_{j=1}^p f_j\cdot 1_A\cdot 1_{B_j}dP=\int \sum_{j=1}^p f_j\cdot 1_{A\cap B_j}dP=\sum_{j=1}^p f_jP(A\cap B_j)=0$, the last equality coming from the fact that we have $P(A)=0$ by hypothesis and therefore it must be $0<=P(A\cap B_j)<=P(A)=0$. We have therefore proved that that for f simple $P(A)=0\implies F=0$.
\\We now consider the case of a positive function that is not simple. We consider $f_n=\sum_{j=1}^{p_n} f_{jn}\cdot 1_{B_jn}$, with $\{f_n\}$ an increasing sequence of simple functions s.t. $f_n\uparrow f$, and therefore $\int_A f_ndP\uparrow\int_A fdP$. We have demonstrated in the first part of this exercise 2.2 that for a simple function, if $P(A)=0$, it will be $\int f_n\cdot 1_{A}dP=\int \sum_{j=1}^{p_n} f_{jn}\cdot 1_A\cdot 1_{B_{jn}}dP=\sum_{j=1}^{p_n} f_jP(A\cap B_{jn})=0$, therefore $\int f_n\cdot 1_{A}dP=0$ $\forall n$, and $0=\lim_{n \to \infty}\int f_n\cdot 1_{A}dP=\int f\cdot 1_{A}dP=F(A)$, so we have demonstrated that $P(A)=0\implies F(A)=0$.
\\\\
\textbf{Ex2.3)}\\
Show that $\int gdP=F=\int fdP\implies f=g$ a.e. ($\Longleftarrow$ is trivial).
\\\\
We can use the result that for $z\geq 0$, $\int zdP=0 \implies z=0$ a.e. We have by hypothesis that $\int (f-g)dP=0$, let $h=f-g$, using $h_+=$max$\{f,0\}$ and $h_-=$min$\{f,0\}$, we have $h=h_+-h_-$. We consider $\Omega_1=\{x\in\Omega \mid f\geq g\}$ and $\Omega_2=\{x\in\Omega \mid f<g\}$, it's $\Omega_1\cap\Omega_2=\varnothing$ and $\Omega_1\cup\Omega_2=\Omega$. Since the property $\int_A hdP=0$ holds on any $A\subseteq\Omega$, we have $\int_{\Omega_1} hdP=0$ and $\int_{\Omega_2} hdP=0$. So, using definitions of $h_+$ and $h_-$, we have $\int_{\Omega_1} hdP=\int_{\Omega_1} h_+dP=0$ and $\int_{\Omega_2} hdP=-\int_{\Omega_2} h_-dP=0\implies\int_{\Omega_2} h_-dP=0$: in both cases we can use the result $z\geq 0$, $\int zdP=0 \implies z=0$ a.e. and therefore we have $f=g$ a.e. on $\Omega_1\cup\Omega_2=\Omega$.
\\\\
\textbf{Ex3.1)}\\
Show that $A_1, A_2,...\in\textbf{\textit{A}}$, $A_1\supset A_2...$ and $P(A_k)\rightarrow 0$ as $k\rightarrow\infty$, then $\int_{A_k}fdp\rightarrow0$ for $k\rightarrow\infty$ and $f\geq 0$.
\\\\
\\We start by proving for simple f. Let $f=\sum_{j=1}^p f_j\cdot 1_{B_j}$, then we have $\int f\cdot 1_{A_k}dP=\int \sum_{j=1}^p f_j\cdot 1_{A_k}\cdot 1_{B_j}dP=\int \sum_{j=1}^p f_j\cdot 1_{A_k\cap B_j}dP=\sum_{j=1}^p f_jP(A_k\cap B_j)$ and we know that $0\leq P(A_k\cap B_j)\leq P(A_k)$, with, by assumption, $P(A_k)\rightarrow 0$ as $k\rightarrow\infty$, therefore we have$\int f\cdot 1_{A_k}dP\rightarrow 0$ as $k\rightarrow\infty$.

\end{enumerate}
\end{document}
