%:
%\documentclass[12pt,mythesisstyle,twoside]{report}%% for two-sided printing
\documentclass[12pt,mythesisstyle]{report}
\usepackage{files/mythesisstyle}
%\usepackage{bbm}

\usepackage{amssymb,amsmath}
\usepackage{calc}
\usepackage{verbatim}
%\usepackage[mathscr]{eucal}
%\usepackage{pstricks}

% \textbf{\textit{A}}

\title{Data And Uncertainity\\Assignment 8 November 2017}
\author{Leonardo Ripoli}
\date{November 2017}

\begin{document}

\begin{titlepage}
\maketitle
\end{titlepage}

Ex. 1) Let \(\Omega=\mathbb{N}\), show that \textbf{\textit{A}}, the set of all cofinite subsets in \(\Omega\), forms an algebra.
\\
\\Proof
\\
1) \(\varnothing\in\textbf{\textit{A}}\), in fact \(\varnothing^\complement=\mathbb{N}\), which is infinite.\\
2) By definition if \(A\in\textbf{\textit{A}}\) it is cofinite, therefore either it is finite and its complement is infinite, or the other way around: in either case \(A\in\textbf{\textit{A}}\implies A^\complement\in\textbf{\textit{A}}\).\\
3) Let \(A_1, A_2, A_3...\in\textbf{\textit{A}}\), then if all of the \(A_i\) are finite, clearly \(\cup_{k=1}^nA_i\) is finite and therefore \(\cup_{k=1}^nA_i\in\textbf{\textit{A}}\). If at least one of the \(A_i\) is infinite, then \(\cup_{k=1}^nA_i\) is infinite, we have to show that \(\cup_{k=1}^nA_i^\complement\) 


\end{document}
